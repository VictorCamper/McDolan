\documentclass[11pt]{utalcaDoc}
\usepackage[activeacute,spanish]{babel}
\usepackage[utf8]{inputenc}
\usepackage{verbatim}
\usepackage[spanish]{babel}
%\usepackage{graphicx}
%\usepackage{latexsym}
%\usepackage{amsmath}
%\usepackage{amssymb}
%\usepackage{amsthm}
%\usepackage{anysize}
%\marginsize{2cm}{2cm}{1.7cm}{1.5cm}
\usepackage[top=2.7cm, bottom=2cm, left=1.8cm, right=1.8cm,headheight=110pt]{geometry}
\usepackage{url}
\usepackage{float}
\usepackage{amsfonts}
\usepackage{listings} % Mostrar código
%\usepackage{algorithmicx}
%\usepackage{algcompatible}
%\usepackage[noend]{algpseudocode}
%\usepackage{algorithm}
%\usepackage{algorithmic}

%\usepackage{listings}

%\usepackage{bussproofs} % pruebas lógicas

\usepackage{fancyhdr}

% aqui definimos el encabezado de las paginas pares e impares.
\lhead[Construcción de Software -- 2017-2]{Construcción de Software -- 2017-2}
%\chead[y1]{y2}
\rhead[Universidad de Talca]{Universidad de Talca}
\renewcommand{\headrulewidth}{0.5pt}



% aqui definimos el pie de pagina de las paginas pares e impares.
%\lfoot[d1]{e1}
%\cfoot[c1]{d2}
\rfoot[Victor Reyes Medina]{Victor Reyes Medina}
\renewcommand{\footrulewidth}{0.5pt}
\pagestyle{fancy} 


% Plantilla inicialmente de rapa, luego le fuí cambiando/agregando algunas cosas

\title{{\bf \Large Construcción de Software \\ 
			Prueba unidad \# 1 -- Parte II}\\ 
			{\normalsize Semestre 2017-2 -- Prof. Daniel Moreno}} 
\author{
  Victor Reyes Medina\\
  \texttt{vireyes14@alumnos.utalca.cl}
}                                               
\date{\today}                                   

\begin{document}
\renewcommand{\figurename}{Figura~}
\renewcommand{\tablename}{Tabla~}
\renewcommand{\lstlistingname}{Código~}
\renewcommand{\theenumii}{\arabic{enumii}}
\renewcommand{\labelenumii}{%
 %\theenumi.\theenumii.
  \theenumii.
}
\maketitle


\section*{a. Camino Crítico}
\subsection*{Tareas para cumplir con las funcionalidades principales}

\begin{tabular}{|c|p{9.5cm}|p{3cm} |c|}
\hline 
\textbf{Tarea} & \textbf{Descripción} & \textbf{Predecesoras} & \textbf{Esfuerzo}\\ 
\hline 
A & Crear nuevas bebidas en el sistema y persistencia de las mismas. & - & 3 días  \\ 
\hline 
B & Agregar bebidas a una cotización. & A & 2 días \\ 
\hline 
C & Crear nuevos toppings en el sistema y persistencia de los mismos. & - & 3 días  \\ 
\hline 
D & Agregar toppings a la cotización de una bebida. & C, B & 1 día  \\ 
\hline 
E & Eliminar Bebidas y Toppings del sistema. & A, C & 1 día  \\ 
\hline 
F & Editar Bebidas y Toppings guardados en el sistema. & A, C & 2 días  \\ 
\hline 
G & Agregar más de una combinación bebida-topping a la cotización & B, D & 1 día  \\ 
\hline 
H & Revisión del código y estandarización del mismo. & A, B, C, D, E, F, G & 2 días  \\ 
\hline 
I & Pruebas. & H & 2 días  \\ 
\hline
J & Instalación del sistema. & I & 1 día  \\ 
\hline



\end{tabular} 

\section*{b. Diseño del Sistema}

\section*{c. Construcción de la aplicación}

\section*{d. Patrones de Diseño}

\end{document}

